\documentclass[12pt]{article}
\usepackage{epsf}
\usepackage{amssymb}
\usepackage{hyperref}
\usepackage[usenames, dvipsnames]{color}

\topmargin      -0.3truein
\oddsidemargin  -0.2truein
\evensidemargin -0.2truein
\textheight     9.5truein
\textwidth      6.5truein
\headheight     0.0truein
\headsep        0.0truein
\parskip 10pt plus 1pt
\parindent=0pt

\begin{document}
\thispagestyle{empty}


\sloppy

\vskip0.3cm
\hrule width11.5cm height1pt % black line
\vskip0.1cm
{\it Tammy Riklin Raviv}\\[0.05cm]
{\small The Department of Electrical and Computer Engineering} \\[0.07cm]
{\small Ben-Gurion University of the Negev} \\
{\small Beer-Shava 84105 {\hskip6pt} ISRAEL}\\[0.1cm]
{\scriptsize Tel: +972 8 642 8812}\hspace{8pt}
               {\scriptsize Fax: +972 8 647 2947}\\
{\scriptsize E-mail: \hspace{1pt} rrtammy@ee.bgu.ac.il}\\

\vspace{-0.7cm}
\hfill April 2018

Editorial Office
Computer Vision and Image Understanding
\vspace{0.2cm}

Dear Prof. Nikos Paragios,


\vspace{0.2cm}

It is our pleasure to submit a new manuscript entitled  ``{\it Accelerated Magnetic Resonance Imaging by Generative Adversarial Neural Networks}'' for
consideration by the Computer Vision and Image Understanding journal. 
The manuscript is co-authored by Ohad Shitrit and Tammy Riklin Raviv.

In this manuscript, we introduce a practical, software-only framework, based on deep learning, for accelerating MRI acquisition, while maintaining anatomically meaningful imaging. The key idea is performing MRI sub-sampling, while using a generative adversarial networks to directly estimate the missing k-space samples.
Comprehensive experiments using brain MRI scans, from a large publicly available dataset, demonstrate the quality of the proposed MRI reconstruction, obtained for up to six-fold acceleration factor. 
Our method is shown to outperform all the other tested methods, including the widely-used Compressed Sensing.


A preliminary version of this work was presented at the Deep Learning in Medical Image Analysis workshop (DLMIA 2017), Quebec, Canada
(in conjunction with the MICCAI conference). The submitted manuscript is substantially
longer and more comprehensive. We further extended the scope of our framework using a more stabilized training regime and exploiting the sparsity in two dimensions. The method is now tested on a large, diverse publicly available dataset of raw images without pre-processing, and with much higher sampling factors. Furthermore, extensive experiments were conducted to demonstrate the clinical applicability of the proposed framework and the advantage of performing the reconstruction directly from the k-space.

Thank you for the consideration of this manuscript. 

We look forward to hearing from you.

\vspace{1cm}


Sincerely,

\vspace{0.2cm}

Tammy Riklin Raviv

\end{document}
